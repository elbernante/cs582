\section{Model Architecture}
In this semantic image segmentation problem, a deep neural network is used to predict the labels for each pixel in the image. Convolutional networks \cite{Lecun98gradient-basedlearning}, from which most state of the art image recognition techniques are derived, will used as the building block of our architecture.

In our dataset, the pixels are roughly similar all throughout the image. There is no clear groupings or sharp boundaries in which an untrained eye can clearly identify the brachial plexus.

The U-Net architecture \cite{2015arXiv150504597R} uses VGG-like layers \cite{2014arXiv1409.1556S}, where a series of convolutional layers are place on top of each other, and gradually decreasing the grid size while increasing the depth. The convolutions are then gradually reversed until the layer is back to its original size. This forms a U-like structure, from which the architecture got its name.

Marko Jocic \cite{jocicmarko-ultrasound-segmentation} has demonstrated the use of U-Net on the same dataset. Our architecture improves U-Net by replacing the VGG-like layers with Inception Modules \cite{2015arXiv151200567S} (See Figure \ref{fig:inception-u-net}). All layers in the middle uses Inception V3 and Grid Reduction modules instead of VGG and MaxPooling. Inception V4 is used in the coarsest layer at the bottom. Transpose Convolution \cite{2016arXiv160307285D} is used to upsample the layer to the same size as the input. The Inception modules made the model very deep and results to vanishing gradients. To solve this, SELU \cite{2017arXiv170602515K} activations are used which make it impossible to have vanishing or exploding gradients. Sigmoid was used at the last layer for binary classification (see Table \ref{tab:comparison}).
\begin{table}[h]
	\centering
	\begin{tabular}{l c c}
	\toprule
	\ & U-Net & Inception U-Net \\
	\midrule
	\midrule
 	Parameters & 7,759,521\footnotemark  & 1,846,353\footnotemark[\value{footnote}] \\
	File size       & 93 MB\footnotemark[\value{footnote}]       & 22.7 MB\footnotemark[\value{footnote}] \\
    Layer           & VGG-like    & Inception Module \\
	Pooling			& Max pooling & Reduction Module \\
    Activations     & ReLU        & SELU \\
	\bottomrule
		
	\end{tabular}
	\caption{Comparison of U-Net and Inception U-Net.}
\label{tab:comparison}
\end{table}

\footnotetext{Based on 96 x 128 input shape}